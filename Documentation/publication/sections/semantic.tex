\section{Overview of OSL}
\label{sec:osl-syntaxandsemantics}

\lstdefinestyle{mystyle}{
    backgroundcolor=\color{backcolour},   
    commentstyle=\color{codegreen},
    keywordstyle=\color{magenta},
    numberstyle=\tiny\color{codegray},
    stringstyle=\color{codepurple},
    basicstyle=\ttfamily\footnotesize,
    breakatwhitespace=false,         
    breaklines=true,                 
    captionpos=b,                    
    keepspaces=true,                 
    numbers=left,                    
    numbersep=5pt,                  
    showspaces=false,                
    showstringspaces=false,
    showtabs=false,                  
    tabsize=2
}


\begin{wrapfigure}{l}{0.4\linewidth}
\centering
\begin{tikzpicture}[node distance=4mm, >=latex',
             block/.style = {draw, rectangle, minimum height=10mm, minimum width=28mm,align=center},
            rblock/.style = {draw, rectangle, rounded corners=0.5em},
]
\node [rblock]                      (src)     {Source program};
\node [block, below=of src]       (oslt)   {Transpilation:\\OSL$_{\Tau}$};
\node [block, below=of oslt]     (osls)  {OSL Operational semantic:\\OSL$_{\mathcal{OS}}$};
\node [block, below=of osls]     (kf)  {${\mathbb{K}}$-Framework};
\node [below=of kf]     (result)  {Satisfy Ownership ? Yes/No};
%% paths (borowed from Harish Kumar)
\path[draw,->] (src)      edge (oslt)
               (oslt)    edge (osls)
               (osls)    edge (kf)
               (kf)    edge (result)
                ;
\end{tikzpicture}
\caption{An overview of the OSL pipeline}
\label{fig:my_label}
\end{wrapfigure}


We present the OSL semantics and syntax. The OSL analyzer is parametric and may be instantiated for a multitude of declarative and imperative programming languages. The OSL pipeline analysis is divided into two parts. The frontend \oslt, take a C program and translates it into an OSL program. We will present \oslt~in  \autoref{ssec:overview-oslt}. The core \oslos~relies on the rewrite-based executable semantics: ${\mathbb{K}}$-Framework, and encodes our ownership semantics of \autoref{sec:ownership}, and verify the soundness of the program translated by \oslt. The composition of  \oslos~and \oslt~allow us to verify the ownership soundness of many C programs. A new programming language can be supported by extending the translator. \\
In the following sections, we first present the \oslos. How this operational semantic allows us to detect memory bugs, and after will we introduce \oslt. 

\subsection{Overview of \oslos }

\subsubsection{Syntax.}

We built OSL$_{\mathcal{OS}}$ on a simple syntax comprising standard composite commands and parametrized by a set of atomic commands. This allows us to instantiate OSL for several use cases without changing the underlying meta-theory. Our syntax  is given by the \autoref{lst:syntax-osl} below, and includes the didactic syntax \autoref{lst:didactic-syntax} presented in \autoref{ssec:Formalisation of core concepts}; with the exception that \term{read} belongs to expression here. \\ 

\begin{lstlisting}[float=ht,caption=OSL$_{\mathcal{OS}}$ syntax, style=osl,label=lst:syntax-osl]
exp   ::= Id | newResource(props) | (*$\star$*) exp
        | call exp (exps) | read(exp) | & exp | &mut exp
exps ::= exp, ..., exp
prop ::= copyable | mutable
props ::= prop, ..., prop
type ::= own(props) | ref(lifetime, type) | #voidTy
lifeftime ::= (*$'Id$*)
stmt ::= exp ; | decl Id ; | exp borrow exp | exp mborrow exp;
        | transfer exp exp ; | free exp; | @ block, ..., block ;
        | ! block ; | val(exp) ; | stmt stmt | block | unsafe block
block ::= { stmt }
function ::= fn Id < annotations > ( parameters ) (*$\rightarrow$*) type block
annotations ::= lifeftime : annotations  | lifetime
            | annotations \/ annotations | annotations /\ annotations
parameters ::= id: type, ..., id: type
PGM ::= stmt
\end{lstlisting}

Similarly, resources are allocated by \term{newResource}($props$) operator, but a new resource can have properties: $props$. We now only consider the property \tagpl{copyable} and \tagpl{mutable}.
The set of properties is extensible to enable OSL to study a larger class of memory properties in the future.
The \term{transfer}, \term{borrow} and \term{mborrow} statements deal mainly in the same way that the corresponding in \osld.
They only vary on the precondition assertions because of the complete configuration of \oslos~change.  
The property \tagpl{copyable} matters here. When a \tagpl{copyable} resource  is transferred  from $e_1$ to $e_2$, a new resource with the same properties is created and its ownership is transferred to $e_2$.
If a resource is not \tagpl{copyable}, then we proceed to a semantic move.
The expressions \term{\& x} and \term{\&mut x} create a reference for $x$.
The opposite of referencing by using \term{\&}  is dereferencing, which is accomplished with the dereference operator \term{$\star$}.
The symbol \term{@} and \term{!} are respectively the branch and loop operators.
A block of code can be prefixed with the \term{unsafe} keyword to permit unsafe operations like creating multiples mutable alias within a safe block.
\subsubsection{\oslos~Model}

\begin{definition}[A \oslos~configuration]
    A \oslos~\textbf{configuration} is defined as a Set of \textbf{state} triple $\cfg{ (\confosl, P, t), CS }$, where $\confosl$ is called an \textbf{environment} tuple $\confoslunfold$, $P$ is program to be executed under environment $\confosl$ and time counter $t$, and we note $CS$ the tail of the Set. A \textbf{configuration} Set his always initialize as a unit-set with triple $\cfg{ \cfg{(\emptyset,\emptyset,\emptyset, \emptyset, \bot), P, 0}}$, where $P$ is the program and the time counter is set at 0.
\end{definition}


Compared with \osld, we use a set of triple state in \oslos~because it supports conditional statement \term{@}. Therefore, to cover all possible execution paths, \oslos will analyze all of them in different states; more details is provided in \autoref{ssec:flow-control}.  The \textbf{environment} tuple  $\confoslunfold$ is composed by the same $\env$ and $\store$ of \osld~ in \autoref{ssec:Formalisation of core concepts}, but we add : \\

\begin{itemize}
    \item[] $\writes: \{ (L, T)  \mid L \in Loc \land T \in \mathbb{N} \}$, is a set recording the writing for locations together with the 
    time instants at which the writings are carried out. $\term{write}(l,t)\in \writes$ denotes a writing operation on $l$ at the time instant $t$.\\

    \item[] $\unsafe:\mathbb{B}$, is a boolean flag that indicate if we enter in unsafe mode. In this mode, some precondition assertions are no longer required. \\
  
    \item[] $\envfun: Id \mapsto (\text{annotations}, \text{parameters}, \text{type}, \text{block})$, a map from $Id$ to a function definition.   
\end{itemize}

Adding $\writes$ will allow us to satisfy the \autoref{def:freeze-owner} and \autoref{def:imm-ref}. For $\unsafe$, to avoid that \oslos~ reject some program because it does not have enough information to be confident. The user can use \term{unsafe} keyword to tell \oslos: \textit{"Trust me, I know what I’m doing."}. The downside is that bugs such as \textit{null pointer dereferencing} can occur in an incorrect code put inside an \term{unsafe} block. Intuitively the map $\envfun$ stores the definition of the function declared and jumps to them when the keyword \term{call} is used.

\subsubsection{Domain.}

The domain of values stored in $\mathcal{S}$ is defined as follows:

\begin{equation*}
    \begin{aligned}
    Value ::&= \#\textbf{rs}(\tagp{props}) \mid  (\mathbb{N}, \mathbb{N}, \textbf{mutRef}(Loc)) \\
    & \mid (\mathbb{N}, \mathbb{N}, \textbf{immRef}(Loc)) \mid \bot \\
    \end{aligned}
\end{equation*}

Compared to \osld~, here resources can be parametrized to indicate if there are \tagp{mutable} or \tagp{copyable}.

\subsection{\oslos~for Detecting memory usage bugs}
\label{ssec:detecting-mem-bugs}

\oslos~has an operational semantic design to detect misuse memory. Violations include use after free, null pointer dereference, using uninitialized memory and double free.
Memory violations like these can cause programs to crash unexpectedly and be exploited to alter intended behavior. Potential consequences of a memory-related bug include information leakage, arbitrary code execution, and remote code execution. According to the statistical report of MITRE \cite{cwe}, these memory-safety bugs are enumerated among the top dangerous software vulnerabilities.
In this section, we will show how the operational semantic of \oslos~ aims to prevent this error. The complete semantic can be found in Appendix \ref{app:complete-oslos}.

\subsubsection{Use after free.}

\begin{wrapfigure}{R}{0.4\linewidth}
\begin{lstlisting}[style=deno]
decl x;
transfer newResource(mut) x;
decl y;
y borrow x;

free(x);
transfer newResource(mut) y;
* y;
\end{lstlisting}
\caption{use after free}
\label{lst:use-after-free}
\end{wrapfigure}

After freeing a pointer, the buffer pointed by the pointer would be deallocated or recycled. However, the pointer still points to the deallocated memory address. These pointers are named dangling pointers. Dereferencing a dangling pointer causes use-after-free issues. Such bugs are dangerous because the freed memory may have already been reallocated for other
purposes, or it may allow users to control the linked list of free memory blocks maintained by the system and achieve arbitrary write \cite{217626}. In the example \autoref{lst:use-after-free}, \term{y} borrow mutably \term{x} at line 5, and then deallocate the resource owned by \term{x}. Consequently, \term{y} become a dangling pointer, and writing to it or dereferencing causes a use-after-free issue. But the semantic rule \autour{Transfer-mutRef} and \autour{Dereference} prevent this violation. The predicate $ write(\term{wv}(e',E''),t_c)$ will not be validated because we can transfer the resource to \term{y} when the lifetime of \term{y} does not intersect with the owner \term{x}. If we omit the \term{transfer}, the predicate $\cfg{ \confosl', \term{read}(l) } \Rightarrow \cfg{ \confosl'', v }$ in \autour{Dereference} will not be validated because \term{x} it is mapping to $\bot$ after the frees.

\begin{figure}
\begin{subfigure}{\textwidth}
    \subcaption{\autour{TrMutRef}}
    \begin{prooftree}
        \label{TrMutRef}
        \hypo{
            \begin{matrix}
                \cfg{ (\confosl,\term{lv}(\tagp{e1}), t) } \Rightarrow \cfg{ (\confosl',l , t)} \\
                l\mapsto_{lf(\_,\_)} \term{mutRef}(l_1)\in S \\
                \confosl''= \confoslunfoldpp \\
                \nexists l_n\mapsto_{lf(t_b,t_e)}\term{anyRef}(l),t_e\geq t_+
            \end{matrix}
        }
        \hypo{
            \begin{matrix}
                t_+ = t + 1 \\
                \cfg{(\confosl',  \term{lv}(\tagb{e2}), t)} \Rightarrow \cfg{(\confosl'',l', t)} \\
                \store'''=\store'' \oplus (l' \mapsto_{lf(t_+,t_+)} \term{mutRef}(l_1)) \oplus (l \mapsto \bot) \\
                \writes'''= \writes''\cup\{write(\term{wv}(\tagb{e2}, \env''),t_+)\}
            \end{matrix}
        }
        \infer2[]{ \cfg{(\confosl, \term{transfer } \tagp{e1}~\tagb{e2};, t), CS} \Rightarrow \cfg{((\env'',\store''',\writes''', \unsafe, \envfun), \emptyset , t_+) , CS} }
    \end{prooftree}
\label{fig:Transfer-mutRef}

\vspace*{0.5cm}
\end{subfigure}

\begin{subfigure}{\textwidth}
    \centering
    \begin{prooftree}
         \hypo{\cfg{ (\confosl,\term{lv}(*e), t)} \Rightarrow \cfg{(\confosl', \tagp{l}, t)} }
         \hypo{\cfg{ (\confosl',\term{read}(\tagp{l}),t)} \Rightarrow \cfg{(\confosl'',\tagp{v},t)} }
        \infer2[\autour{Dereference}]{ \cfg{ (\confosl, * e, t), CS } \Rightarrow  \cfg{ (\confosl'' , \tagp{v}, t), CS } }
    \end{prooftree}
    \label{fig:deref-rule}
\end{subfigure}
\end{figure}


\subsubsection{Double free.}



\begin{wrapfigure}{r}{0.4\linewidth}
\begin{lstlisting}[style=deno]
decl x;
transfer newResource(mut) x;
free(x);
free(x);
\end{lstlisting}
\caption{Double free}
\label{lst:double-free}
\vspace{-70pt}
\end{wrapfigure}

Freeing a pointer twice causes double free issues. It leaves room for attackers to manipulate the free memory
blocks. Consequently, attackers may overwrite any memory cell of the process leveraging memory allocation system calls.\\\\
In the example \autoref{lst:double-free}, the \term{free} at line 4 will not be validated because \term{x} maps to $\bot$ after the previous \term{free}. So the predicate $\nexists~l \mapsto \bot \in \store$ in \autour{Free} is not valid and \oslos~will return that the program is unsound.

\vspace{0.2cm}
\begin{prooftree}
    \vspace{\baselineskip}
    \hypo{
        \begin{matrix}
            \confosl = \confoslunfold \\
            \cfg{ (\confosl, \term{lv}(x),t) } \Rightarrow \cfg{ (\confosl, l, t) } 
        \end{matrix}
    }
    \hypo{ \tagp{$\store'$} = \store \oplus (l \mapsto \bot ) }
    \hypo{ \nexists~l \mapsto \bot \in \store }
    \infer3[\autour{Free}]{ \cfg{ (\confosl, \term{free}(x), t), CS } \Rightarrow { \cfg{ ((\env, \tagp{$\store'$},\writes, \unsafe, \envfun), \emptyset,t+1), CS } } }
\end{prooftree}


\subsubsection{Uninitialized memory access.}

\begin{wrapfigure}{r}{0.4\linewidth}
    \begin{lstlisting}[style=deno]
    decl x;
    decl y;
    decl z;
    read(x);
    y borrow x;
    z mborrow x;
    read(y);
    read(z);
    \end{lstlisting}
\caption{Uninitialized access}
\label{lst:uninit-mem}
\end{wrapfigure}

It means the memory is used without proper initialization.
Intuitively, it may leak the old content. If the memory contains values of pointer types, falsely accessing or free the pointer would cause invalid memory access. In the rules \autour{Read-Rs}, \autour{Read-immRef} and \autour{Read-mutRef}, all are protected by a predicate from uninitialized memory access. If we try to apply \autour{Read-Rs} at line 2, then the predicate $l \mapsto \#rs(ps) \in S$ will not be satisfy with \term{x}. The variable \term{x} is still mapping to $\bot$, because we did not \term{transfer} a resource. 
Let's analyze another case  when we try to apply \autour{Read-Rs} or \autour{Read-Rs}. The common predicate $\cfg{(\env,\store', \dots),\term{read}(l')} \Rightarrow^+ \cfg{(\env,\store'', \dots),\#rs(\_)}$ where \term{x} is $l'$,  will not be validated. Indeed 
both \term{y} and \term{z} borrow the owner \term{x}, but \term{x} is uninitialized which violate the previous common predicate that requires an initialized owner.

\begin{figure}[H]
    % READ RS ==============
    \begin{subfigure}{\textwidth}
        \centering
        \subcaption{\autour{Read-Rs}}
        \begin{prooftree}
            \hypo{\tagp{l} \mapsto \tagb{\#rs(ps)} \in \store}
            \infer1[]{ \cfg{ (\confoslunfold, \term{read}(\tagp{l}), t), CS} \Rightarrow  \cfg{ (\confoslunfold, \tagb{\#rs(ps)},t), CS }}
        \end{prooftree}
        \label{fig:read-rs}
        \vspace{0.5cm}
    \end{subfigure}
    % READ SHR REF ============
    \begin{subfigure}{\textwidth}
        \centering
        \subcaption{\autour{Read-immRef}}
        \begin{prooftree}
         \hypo{\begin{matrix}
            l \mapsto_{lf(t_b,t_e)} \term{immRef}(l')\in \store \\
            \store' = \store \oplus [l \mapsto_{lf(t_b,t_+)}\term{immRef}(l')] \\
            \cfg{ ((\env, \store', \writes,\unsafe,\envfun), \term{read}(l'), t)} \Rightarrow^+ \cfg{(\confosl'' ,\#rs(\_), t)} \\
            \nexists l_1\mapsto_{lf(t_b',t_e')} \term{mutRef}(l')\in S, l\neq\l_1 \\
         \end{matrix}}
         \hypo{\begin{matrix}
            \neg\term{wl}(l',t_b,t_+,\writes) \\
            \term{lc}(t_b,t_+,t_b',t_e') \\
            t_+= t + 1 \\
         \end{matrix}}
        \infer2[]{ \cfg{ ( \confoslunfold , \term{read}(l), t),CS } \Rightarrow  \cfg{ (\confosl'', l', t_+), CS } }
        \end{prooftree}
    \end{subfigure}
    \begin{subfigure}{\textwidth}
        \centering
        \begin{align*}
            & \textbf{lc}(t_1,t_2,t_3,t_4)\triangleq (t_1< t_3\leq t_2)\lor (t_1\leq t_4\leq t_2) \\
            & \textbf{wl}(l,t,t',W)\triangleq \exists write(l,t'')\in W \land t \leq t'' \leq t'
        \end{align*}        
    \label{fig:read-rs}
    \end{subfigure}

    \label{fig:Read-immRef}
\end{figure}

\subsection{Overview of \oslt}
\label{ssec:overview-oslt}

We next present the operational semantics of \oslt. It automatically translates C programs into the mid-level intermediate representation (MIR) OSL syntax.
Currently, we support the complete  C11  standard version syntax and programs that do not include \textbf{goto} statement. However, we aim to support the translation of any C program.


\subsubsection{\oslt~States.} A \oslt~\textit{state} is a Triple $\cfg{P , \map, \out }$, where $P$ is a list of C tokens,
$\map$ is a map between \textit{Identifier} (variable, function name) to \oslos~\term{Type}, and $\out$ is a Monoid ($O$, $\append$) where $O$ is a \oslos~ program, the associative binary operation $\append$ is the \textit{append} function and the \textit{zero} element $\epsilon$ represent the empty list.
The \textit{triple} $(P, \emptyset, \epsilon)$ is the initial \textit{triple} of \oslt.


\subsubsection{Transpilation of C program.}  The transition between Triple is denoted by $\Rightarrow$. For C expression, the final value is the corresponding translated expression; for C Statement, it is the translated statement with the $\map$ enriched with new variables and functions declared in $P$.

\begin{align*}
&\evale ~::  ( Exp \times Env ) \rightarrow Output \\
&\evals ~::  ( Stmt \times Env \times Output ) \rightarrow Env \times Output \\
\end{align*}
\noindent\textbf{Notation.} We use $DS$ as \textit{Declaration-Specifier} from C11 \cite{ISOC11} standard, which contains the \textit{Type-Specifier} (e.g.: \term{int}, \term{float}) and \textit{Type-Qualifier} (e.g.: \term{const}, \term{atomic}) information. We write $DD$ as \textit{Derived-Declarator} which tell us if the declarator is a pointer, array or a function. We continue to use the notation $\oplus$ as modification operation on a Map.\\

\noindent We infer the properties (\textit{props}) of a variable by looking at his type signature composed by a tuple ($DS$, $DD$). 
The function $\Tau$ take a $DS$ and a $DD$ and return the corresponding \oslos \textit{Type}.  

\begin{align*}
    \Tau (DS, DD) & \rightharpoonup \text{Type}  \\
    \Tau(T, \emptyset) & \rightarrow  \#own(\#rs(\tagpl{mutable}, copyable)) \\
    & \textbf{ with } T \in \{ int,.., float \} \\
    \Tau(const~T, \emptyset) & \rightarrow  \#own(\props{T} \setminus \tagpl{mutable})) \\
    \Tau(T, \star) & \rightarrow  \#ref(\textit{nl}, \Tau(T)) \\
    \Tau(const~T, \star) | \Tau(const~T, const~\star)  & \rightarrow  \#ref(\textit{nl}, \Tau(T)) \\
    & \textbf{ with }   \tagpl{mutable} \notin \props{\Tau(T)}  \\
    \Tau(void, \emptyset) & \rightarrow  \#own(\#rs()) \\
    \Tau(void, \star) & \rightarrow  \#ref(\textit{nl},\#rs()) \\
    \Tau(\term{enum}, \star) & \rightarrow  \#ref(\textit{nl},\#own(\#rs(copyable, \tagpl{mutable}))) \\
    \Tau(const~\term{enum}, \star) & \rightarrow  \#ref(\textit{nl},\#own(\#rs())) \\
    \Tau(\term{struct} \{ m_1\dots m_n \}, \emptyset) & \rightarrow  \#own(\#rs(ps)) \\
                                                      & \textbf{ with } ps = \bigcap\limits_{i=1}^{n} \pi_{props}(\Tau(m_i)) \\
\end{align*}

\begin{equation*}
    \begin{aligned}[c]
        \mathcal{R} (\text{DS, DC}) & \rightharpoonup \text{Type}  \\
        \mathcal{R}(void) & \rightarrow \#voidTy \\
        \mathcal{R}(T) & \rightarrow \Tau(T) \\
    \end{aligned}
\qquad\qquad
    \begin{aligned}[c]
        \pi_{props}(Type) & \rightharpoonup Props \\
        \pi_{props}(\#rs(ps)) & \rightarrow ps \\
        \pi_{props}(\#ref(l, T)) & \rightarrow \pi(T)
    \end{aligned}
\end{equation*}

\oslos~ does not support yet \textbf{partial borrowing} that allow borrowing disjoint fields of a composite structure simultaneously. All access to a member is reduced to access the root.
The properties of a composite struct inferred by $\Tau$ are the intersection of all the members' properties. The function R infers the return type of a C function drawing on $\Tau$.
The function $\pi_{props}$ is a projection function that returns the properties of a \oslos~\textit{Type}.

\subsubsection{\oslt~semantic at a glance.} We provide an overview of the \oslt~semantic rules. The rest of the semantics can be found in Appendix \ref{app:complete-oslt}. In \refrule{Decl}, we translate a C declaration statement.
We first evaluate the properties of \textit{type specifiers} to store them in $\map$ and then append the statement $\term{decl } x$ to $\out$. In \refrule{Ift}, first, we translate the condition and the two endowed statements and then append them to $\out$ respecting the order. We insert a copy of the condition translated at the statement to simulate the looping verification condition. Finally, the rule \refrule{Brw} present the semantics of translating an immutable borrow. We have an immutable borrow when the address $\&y$ assigned to $x$ points to a variable $y$ that does not have the property \tagpl{mutable}.


\begin{figure}[H]
    \begin{subfigure}{\textwidth}
        \centering
        \label{Decl}
        \begin{prooftree}
        \infer0[\autour{Decl}]{
            \cfg{\{\term{T}~x ; \}, \map, \out  }
            \evals
            \cfg{ \map \append [x \mapsto \Tau(\term{T})], \out \append  \term{decl}~x; } 
        }
        \end{prooftree}
        \vspace{\baselineskip}
    \end{subfigure}
    \begin{subfigure}{\textwidth}
        \centering
        \label{Ift}
        \begin{prooftree}
        \hypo{ \cfg{c, \map } \evale \out_c }
        \hypo{ \cfg{b_1, \map, \out} \evals \cfg{ \map_1, \out_1  } }
        \hypo{ \cfg{b_2, \map, \out} \evals \cfg{ \map_2, \out_2 } }
        \infer3[\autour{IfT}]{
            \cfg{\text{\term{if} (c) $b_1$ \term{else} $b_2$}, \map, \out }
            \evals
            \cfg{\map, \out \append \out_c \append \term{@ \{}  \out_1 \term{\}} \append \term{\{} \out_2 \term{\}}  }
        }
        \end{prooftree}
        \vspace{\baselineskip}
    \end{subfigure}
    % Borrow  
    \begin{subfigure}{\textwidth}
        \centering
        \label{Brw}
        \begin{prooftree}
        \hypo{\map[y] = \#rs(ps) }
        \hypo{ \tagpl{mutable} \notin ps }
        \infer2[\autour{Brw}]{
            \cfg{\text{x = \&y}, \map}
            \evale
            x~\term{borrow}~y 
        }
    \end{prooftree}
    %  Borrow Mut
    \end{subfigure}
\end{figure}

\subsubsection{Transpilation example.} \autoref{lst:chttp} and \autoref{lst:chttptranspiled} shows an example of \oslt~in action.
\autoref{lst:chttp} is a fragment of a tiny C HTTP server program.\\
 
\begin{lstlisting}[style=clang,caption={C Tiny-HTTP Server},label=lst:chttp]
#define PORT 8080
int main() {
    int server_fd, new_socket; long value_read;
    struct sockaddr_in address; int addrlen = sizeof address;
    char hello[73] = "HTTP/1.1 200 OK\nContent-Type: text/plain\nContent-Length: 12\n\nHello world!";
    server_fd = socket(AF_INET, SOCK_STREAM, 0);

    address.sin_family = AF_INET;
    address.sin_addr.s_addr = INADDR_ANY;
    address.sin_port = htons( PORT );

    memset(&address.sin_zero, '\0', sizeof address.sin_zero);    
    bind(server_fd, &address, sizeof address);
    listen(server_fd, 10);

    while(1)
    {
        new_socket = accept(server_fd, &address, &addrlen);
        char buffer[30000] = {0};
        value_read = read( new_socket , &buffer, 30000);
        write(new_socket , &hello , strlen(&hello));
        close(new_socket);
    }
    return 0;
}
\end{lstlisting}

\autoref{lst:chttptranspiled} is the corresponding \oslt~translation output for \autoref{lst:chttp} main function.\\

\begin{lstlisting}[style=osl,caption={OSL Corresponding transpilation},label=lst:chttptranspiled]
fn main() -> #own(copyable,mutable) {
    decl server_fd; decl new_socket;
    decl value_read; decl address;
    decl addrlen;  decl hello;
    transfer newResource(copyable,mutable) addrlen;
    transfer newResource(mutable) hello;
    transfer call socket(newResource(copyable,mutable),newResource(copyable,mutable),newResource(copyable,mutable)) server_fd;
    rd(server_fd);
    transfer newResource(copyable,mutable) address;
    transfer newResource(copyable,mutable) address;
    transfer call htons(newResource(copyable,mutable)) address;
    call memset(&mut address,newResource(copyable,mutable),newResource(copyable,mutable));

    call bind(server_fd, &mut address, newResource(copyable,mutable));
    call listen(server_fd, newResource(copyable,mutable));
    !{
        transfer call accept(server_fd, &mut address, &mut addrlen) new_socket;
        rd(new_socket);

        decl buffer;
        transfer newResource(mutable) buffer;
        transfer call read(new_socket, &mut buffer, newResource(copyable,mutable)) value_read;
        call write(new_socket, &hello, call strlen(&hello));
        call close(new_socket);
        
    };
    val(newResource(copy,mutable))
};
call main();
\end{lstlisting}

As we do not support \textbf{partial borrowing} yet, the assignments to \term{adress} members (sin\_family, sin\_family, sin\_port) between lines 8-10  in \autoref{lst:chttp} are treated as assignment to the parent: \term{address}. Therefore, in \autoref{lst:chttptranspiled}, we \term{transfer} a new resource to \term{address} three times at the lines 11-13.